\addcontentsline{toc}{section}{СПИСОК ИСПОЛЬЗОВАННЫХ ИСТОЧНИКОВ}

\begin{thebibliography}{9}

    \bibitem{javascript} Шилдт Герберт. Java полное руководство / Герберт Шилдт. – М : ООО ”И.Д. Вильяме 2015. – 1376 с. - ISBN 978-5-84-591759-1.
    \bibitem{php} Коэн Исси, Лазаро; Исси Коэн, Джозеф. Полный справочник по HTML, CSS и JavaScript / А.О. Коэн Исси, Лазаро; Исси Коэн, Джозеф. – Паблишерз : Эксмо, 2017. – 246 с. - ISBN 978-5-9790-0009-1.
    \bibitem{css} Хорстманн, К. Современный JavaScript для нетерпеливых / К. Хорстманн; перевод с английского А. А. Слинкина. — Москва: ДМК Пресс, 2021. — 288 с. — ISBN 978-5-97060-177-8.
    \bibitem{mysql}	Мандел, Т. Разработка пользовательского интерфейса / Т. Мандел. – ДМК Пресс, 2019. – 420 с. – ISBN 978-5-04-195060-6.
	\bibitem{html5}	Купер А., Рейман Р., Кронин Д., Носсел К. Интерфейс. Основы проектирования взаимодействия : [Текст] / А. Купер, Р. Рейман, Д. Кронин, К. Носсел; пер. с англ. – 4-е изд. – СПб. : Питер, 2021. – 720 с. – ISBN 978-5-4461-0877-0.
	\bibitem{htmlcss}	Джон Карнелл, Иллари Уайлупо Санчес. Микросервисы Spring в действии / пер. с англ. А. Н. Киселева. – М.: ДМК Пресс, 2022. – 490 с. - ISBN 978-5-97060-971-2.
	\bibitem{bigbook} Кугушева Дарья Сергеевна. Проектирование сложного программного обеспечения с использованием микросервисной архитектуры // Инновации и инвестиции. 2020. №5. URL: https://cyberleninka.ru/article/n/proektirovanie-slozhnogo-programmnogo-obespecheniya-s-ispolzovaniem-mikroservisnoy-arhitektury (дата обращения: 27.04.2024).
	\bibitem{uchiru} Поллард Б. HTTP/2 в действии / Б. Поллард. – Москва : ДМК Пресс, 2021. – 424 с. – ISBN 978-5-97060-925-5.
	\bibitem{chaynik}	Бауэр К., Кинг Г. Java Persistence API и Hibernate / К. Бауэр, Г. Кинг. – Москва : ДМК Пресс, 2018. – 632 с. – ISBN 978-5-97060-674-2.   
	\bibitem{22} Стойка А. Учебник по React: современное руководство / А. Стойка. – СПб: Питер, 2023. – 350 с. – ISBN 978-5-00146-705-2.    
	\bibitem{1231} Ли Дж. Руководство по React для начинающих / Дж. Ли. – М.: ДМК Пресс, 2022. – 280 с. – ISBN 978-5-97060-774-9.    
	\bibitem{sdf} Робинсон А. Express.js в действии / А. Робинсон. – М.: ДМК Пресс, 2021. – 320 с. – ISBN 978-5-97060-854-8.    
	\bibitem{servsssds} Вилсон М. Node.js для профессионалов / М. Вилсон. – СПб: Питер, 2022. – 450 с. – ISBN 978-5-00116-800-3.
	\bibitem{111} Ньюман С. Построение микросервисов / С. Ньюман. – СПб: Питер, 2020. – 360 с. – ISBN 978-5-4461-0870-1.
	\bibitem{1112} Фаулер М. Архитектура корпоративных приложений / М. Фаулер. – М.: ДМК Пресс, 2019. – 520 с. – ISBN 978-5-97060-775-6.
	\bibitem{1113} Браун Д. RESTful API Design / Д. Браун. – М.: ДМК Пресс, 2019. – 400 с. – ISBN 978-5-97060-861-6.
	\bibitem{1114} Рэмси М. Использование Axios с React.js / М. Рэмси. – М.: ДМК Пресс, 2023. – 260 с. – ISBN 978-5-97060-964-4.
	\bibitem{1115} Мессерсмит Дж. DevOps для разработки программного обеспечения / Дж. Мессерсмит. – М.: ДМК Пресс, 2021. – 370 с. – ISBN 978-5-97060-913-2.
	\bibitem{1116} Гранд Р. Полное руководство по тестированию программного обеспечения / Р. Гранд. – СПб: Питер, 2022. – 420 с. – ISBN 978-5-00117-895-8.
	\bibitem{1117} Фриман Э. Архитектура программного обеспечения / Э. Фриман. – М.: ДМК Пресс, 2021. – 480 с. – ISBN 978-5-97060-973-6.
	\bibitem{1118} Мартин Р. Чистый код: создание, анализ и рефакторинг / Р. Мартин. – М.: Питер, 2020. – 464 с. – ISBN 978-5-496-00667-5.
\end{thebibliography}
