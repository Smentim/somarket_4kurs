\section*{ВВЕДЕНИЕ}
\addcontentsline{toc}{section}{ВВЕДЕНИЕ}

В мире массовых многопользовательских онлайн-игр жанра MMORPG, таких, как «Stay Out», игроки могут взаимодействовать в виртуальном мире, выполнять квесты, сражаться с монстрами, развивать своих персонажей и, конечно же, торговать внутри игровыми ценностями. MMORPG представляют собой уникальный игровой жанр, где тысячи игроков со всего мира могут взаимодействовать между собой в одном виртуальном мире, создавая атмосферу постоянного развития и приключений.

Сервис для обмена и продажи виртуальных ценностей в MMORPG
призван упростить процесс торговли между игроками, предоставляя им удобную платформу для взаимодействия. Необходимость в таком сервисе возникает из-за того, что в некоторых играх обмен виртуальными ценностями может быть неудобным и невыгодным для игроков. Это может приводить к созданию различных групп в социальных сетях и мессенджерах, где игроки
выкладывают свои товары и договариваются о сделках.

Целью таких сервисов является упрощение процесса торговли, делая
его более удобным и безопасным для всех участников. Путем создания специализированной платформы для обмена и продажи виртуальных ценностей, игрокам будет легче находить нужные товары, договариваться о цене и завершать сделки, минимизируя риск обмана и недобросовестных сделок.

Преимущества сервиса для обмена виртуальными ценностями:
\begin{itemize}
	\item сервис не требует значительных ресурсов для использования и позволяет игрокам свободно обмениваться своими предметами;
	\item пользователи могут легко связываться друг с другом, договариваться о цене и условиях сделки, что делает процесс торговли более удобным и эффективным;
	\item  в сервисах для обмена виртуальными ценностями нет лишней административной информации и правил, устанавливаемых владельцем группы или сообщества, что делает процесс торговли более прозрачным и справедливым для всех участников.
\end{itemize}

\emph{Цель настоящей работы} – разработка web-сайта для упрощения торговли в игре «Stay Out», ввиду отсутствия каких-либо хороших альтернатив. Для достижения поставленной цели необходимо решить \emph{следующие задачи:}
\begin{itemize}
	\item провести анализ предметной области;
	\item разработать концептуальную модель программной системы;
	\item спроектировать и реализовать серверную часть программной системы;
	\item спроектировать и реализовать клиентскую часть программной системы;
	\item провести тестирование работы программной системы.
\end{itemize}

\emph{Структура и объем работы.} Отчет состоит из введения, 4 разделов основной части, заключения, списка использованных источников, 2 приложений. Текст выпускной квалификационной работы равен \formbytotal{lastpage}{страниц}{е}{ам}{ам}.

\emph{Во введении} сформулирована цель работы, поставлены задачи разработки, описана структура работы, приведено краткое содержание каждого из разделов.

\emph{В первом разделе} на стадии описания технической характеристики предметной области приводится сбор информации о деятельности компании, для которой осуществляется разработка сайта.

\emph{Во втором разделе} на стадии технического задания приводятся требования к разрабатываемому сайту.

\emph{В третьем разделе} на стадии технического проектирования представлены проектные решения для web-сайта.

\emph{В четвертом разделе} приводится список классов и их методов, использованных при разработке сайта, производится тестирование разработанного сайта.

В заключении излагаются основные результаты работы, полученные в ходе разработки.

В приложении А представлен графический материал.
В приложении Б представлены фрагменты исходного кода. 
