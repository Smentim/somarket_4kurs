\appendix{Фрагменты исходного кода программы}
ClassController.js
\begin{lstlisting}[language=C++]
	const { Class } = require("../models/models")
	
	class ClassController {
		
		async create(req, res) {
			try {
				let {name, typeId} = req.body
				const clas = await Class.create({name, typeId})
				return res.json(clas)
			} catch (error) {
				console.error("Error creating class:", error);
				return res.status(500).json({ message: "Internal server error" });
			}
		}
		
		async getAll(req, res) {
			const clases = await Class.findAll()
			return res.json(clases)
		}
	}
	
	module.exports = new ClassController()
\end{lstlisting}

itemController.js
\begin{lstlisting}[language=C++]
	const uuid = require('uuid')
	const path = require('path')
	const {Item, AmmoInfo, Equipment, Weapon, Containers, Artefacts, Grenades, StorageItem, User, Type, Class} = require('../models/models')
	const ApiError = require('../error/ApiError')
	const { Op } = require('sequelize')
	
	class ItemController {
		
		async create(req, res, next) {
			try {
				let {name, description, weight, level, typeId, classId, ammo_infos, artefacts, equipment, weapons, containers, grenade} = req.body
				const {img} = req.files
				let fileName = uuid.v4() + ".png"
				img.mv(path.resolve(__dirname, '..', 'static', fileName))
				
				const item = await Item.create({name, description, weight, level, typeId, classId, img: fileName})
				
				if (ammo_infos) {
					ammo_infos = JSON.parse(ammo_infos);
					ammo_infos.forEach(async i => {
						await AmmoInfo.create({
							type_ammo: i.type_ammo,
							breaking_throught: i.breaking_throught,
							damage: i.damage,
							itemId: item.id
						});
					});
				}
				
				if (weapons) {
					weapons = JSON.parse(weapons);
					weapons.forEach(async i => {
						await Weapon.create({
							moa: i.moa, 
							rate_of_fire: i.rate_of_fire, 
							breaking_throught: i.breaking_throught, 
							strength: i.strength, 
							recoil: i.recoil,
							handing: i.handing,
							state_waivers: i.state_waivers,
							nopollution: i.nopollution,
							itemId: item.id
						})
					})
				}
				
				return res.json(item)
			} catch (e) {
				next(ApiError.badRequest(e.message))
			}
		}
		
		async getAll(req, res) {
			let {classId, typeId, limit, page} = req.query
			page = page || 1
			limit = limit || 15
			let offset = page * limit - limit
			let items;
			if (!classId && !typeId) {
				items = await Item.findAndCountAll({limit, offset})
			}
			if (classId && !typeId) {
				items = await Item.findAndCountAll({where:{classId}, limit, offset})
			}
			if(!classId && typeId) {
				items = await Item.findAndCountAll({where:{typeId}, limit, offset})
			}
			if (classId && typeId) {
				items = await Item.findAndCountAll({where:{classId, typeId}, limit, offset})
			}
			return res.json(items)
		}
		
		async getOne(req, res) {
			const {id} = req.params
			const item = await Item.findOne(
			{
				where: {id},
				include: [{model: AmmoInfo, as: 'ammo_infos'}]
			}
			)
			return res.json(item)
		}
		
		async search(req, res) {
			const { name } = req.params;
			
			if (!name) {
				return res.status(400).json({ message: 'Название предмета не может быть пустым' });
			}
			
			try {
				const items = await Item.findAll({
					where: {
						name: {
							[Op.like]: `%${name}%`
						}
					}
				});
				return res.json(items);
			} catch (error) {
				console.error('Ошибка при поиске предмета:', error);
				return res.status(500).json({ message: 'Ошибка при поиске предмета' });
			}
		}
		
		async getStorageItems(req, res, next) {
			const { server } = req.params
			let { page, limit, classId, typeId } = req.query;
			page = page || 1
			limit = limit || 15
			let offset = page * limit - limit
			
			const where = {};
			if (classId) where.classId = classId;
			if (typeId) where.typeId = typeId;
			
			try {
				const items = await StorageItem.findAndCountAll({
					where: { server },
					include: [
					{
						model: Item,
						attributes: ['id', 'name', 'img']
					},
					{
						model: User,
						attributes: ['id', 'email', 'ru1', 'ru2', 'ru3', 'eu1', 'us1']
					}
					],
					limit,
					offset,
					order: [
					['itemId', 'ASC'],
					['userId', 'ASC']
					]
				});
				
				return res.json(items);
			} catch (error) {
				console.error("Ошибка при получении предметов из хранилища:", error); // Логируем ошибку
				next(ApiError.internal(error.message));
			}
		}
		
		async getUserItems(req, res, next) {
			const { userId } = req.query;
			
			try {
				const items = await StorageItem.findAll({
					where: { userId },
					include: [
					{
						model: Item,
						attributes: ['id', 'name', 'img']
					},
					{
						model: User,
						attributes: ['id', 'email']
					}
					],
					order: [['id', 'ASC']]
				});
				
				return res.json(items);
			} catch (error) {
				console.error("Ошибка при получении предметов пользователя:", error);
				next(ApiError.internal(error.message));
			}
		}
		
		async addItemToStorage(req, res, next) {
			const { userId, itemId, price, quantity, classId, typeId, server } = req.body;
			try {
				// Создаем запись в StorageItem
				const storageItem = await StorageItem.create({
					userId,
					itemId,
					price,
					quantity,
					classId,
					typeId,
					server
				});
				
				return res.json(storageItem);
			} catch (error) {
				return next(error);
			}
		}
		
		async deleteItem(req, res, next) {
			try {
				const { id } = req.params;
				const item = await StorageItem.findOne({ where: { id } });
				
				if (!item) {
					return next(ApiError.badRequest('Предмет не найден'));
				}
				
				// Удаление предмета из базы данных
				await item.destroy();
				
				return res.json({ message: 'Предмет успешно удален' });
			} catch (error) {
				console.error('Ошибка при удалении предмета:', error);
				next(ApiError.internal('Ошибка при удалении предмета'));
			}
		}
		
		async getAmmoInfosByItemId(req, res, next) {
			const { itemId } = req.params;
			try {
				const ammoInfos = await AmmoInfo.findAll({ where: { itemId } });
				return res.json(ammoInfos);
			} catch (error) {
				console.error('Error fetching ammo_infos:', error);
				return res.status(500).json({ error: 'Internal server error' });
			}
		}
	}
	
	module.exports = new ItemController()
\end{lstlisting}
	
typeController.js
\begin{lstlisting}[language=C++]
	const {Type} = require('../models/models');
	const ApiError = require('../error/ApiError');
	
	class TypeController {
		
		async create(req, res) {
			const {name} = req.body
			const type = await Type.create({name})
			return res.json(type)
		}
		
		async getAll(req, res) {
			const types = await Type.findAll()
			return res.json(types)
		}
	}
	
	module.exports = new TypeController()
\end{lstlisting}

userController.js
\begin{lstlisting}[language=C++]
	const ApiError = require('../error/ApiError');
	const bcrypt = require('bcrypt')
	const jwt = require('jsonwebtoken')
	const {User, Storage, StorageItem, Item} = require('../models/models')
	
	const generateJwt = (id, email, role, ru1, ru2, ru3, eu1, us1) => {
		return jwt.sign(
		{id: id, email, role, ru1, ru2, ru3, eu1, us1}, 
		process.env.SECRET_KEY,
		{expiresIn: '24h'}
		)
	}
	
	class UserController {
		
		async registration(req, res, next) {
			const {email, password, role} = req.body
			if (!email || !password) {
				return next(ApiError.badRequest('Некорректный email или password'))
			}
			const candidate = await User.findOne({where: {email}})
			if (candidate) {
				return next(ApiError.badRequest('Пользователь с таким email уже существует'))
			}
			const hashPassword = await bcrypt.hash(password, 5)
			const user = await User.create({email, role, password: hashPassword})
			const basket = await Storage.create({userId: user.id})
			const token = generateJwt(user.id, user.email, user.role, user.ru1, user.ru2, user.ru3, user.eu1, user.us1)
			return res.json({token})
		}
		
		async login(req, res, next) {
			const {email, password} = req.body
			const user = await User.findOne({where: {email}})
			if (!user) {
				return next(ApiError.internal('Пользователь с таким именем не найден'))
			}
			let comparePassword = bcrypt.compareSync(password, user.password)
			if (!comparePassword) {
				return next(ApiError.internal('Указан неверный пароль'))
			}
			const token = generateJwt(user.id, user.email, user.role, user.ru1, user.ru2, user.ru3, user.eu1, user.us1)
			return res.json({token})
		}
		
		async check(req, res, next) {
			const token = generateJwt(req.user.id, req.user.email, req.user.role, req.user.ru1, req.user.ru2, req.user.ru3, req.user.eu1, req.user.us1)
			return res.json({token})
		}
		
		async changePassword(req, res, next) {
			const { currentPassword, newPassword } = req.body;
			const userId = req.user.id;
			
			const user = await User.findOne({ where: { id: userId } });
			if (!user) {
				return next(ApiError.badRequest('Пользователь не найден'));
			}
			
			const isPasswordValid = await bcrypt.compare(currentPassword, user.password);
			if (!isPasswordValid) {
				return next(ApiError.badRequest('Неверный текущий пароль'));
			}
			
			const hashPassword = await bcrypt.hash(newPassword, 5);
			user.password = hashPassword;
			await user.save();
			
			return res.json({ message: 'Пароль успешно изменен' });
		}
		
		async createStorage(req, res, next) {
			const { userId } = req.body;
			const storage = await Storage.create({ userId });
			return res.json(storage);
		}
		
		async addItemToStorage(req, res, next) {
			const { storageId, itemId, price, quantity } = req.body;
			const storageItem = await StorageItem.create({ storageId, itemId, price, quantity });
			return res.json(storageItem);
		}
		
		async getItemsFromStorage(req, res, next) {
			const { storageId, classId, typeId, limit, page } = req.query;
			const items = await StorageItem.findAll({
				where: { storageId },
				include: [
				{
					model: Item,
					where: {
						...(classId && { classId }),
						...(typeId && { typeId })
					}
				}
				],
				limit,
				offset: (page - 1) * limit
			});
			return res.json(items);
		}
		
		async updateNicknames(req, res, next) {
			const { userId, nicknames } = req.body;
			
			// Логирование входящих данных
			console.log("updateNicknames - userId:", userId);
			console.log("updateNicknames - nicknames:", nicknames);
			
			if (!userId) {
				return next(ApiError.badRequest('User ID is required'));
			}
			
			try {
				const user = await User.findByPk(userId);
				
				if (!user) {
					return next(ApiError.notFound('User not found'));
				}
				
				user.ru3 = nicknames.ru3 || user.ru3;
				user.ru2 = nicknames.ru2 || user.ru2;
				user.ru1 = nicknames.ru1 || user.ru1;
				user.eu1 = nicknames.eu1 || user.eu1;
				user.us1 = nicknames.us1 || user.us1;
				
				await user.save();
				
				return res.json({ message: 'Nicknames updated successfully' });
			} catch (error) {
				console.error('Error updating nicknames:', error);
				return next(ApiError.internal(error.message));
			}
		}
	}
	
	module.exports = new UserController()
\end{lstlisting}

Main.js
\begin{lstlisting}[language=C++]
	import React, { useState } from "react";
	import { Button, Container } from "react-bootstrap";
	import ServerSelectModal from "../components/ServerSelectModal";
	
	const Main = () => {
		const [modalShow, setModalShow] = useState(false)
		return (
		<div className="home-background">
		<Container className="text-center text-white d-flex flex-column justify-content-center align-items-center vh-100">
		<h1>SO MARKET</h1>
		<h2>УДОБНАЯ ПЛОЩАДКА ДЛЯ ТОРГОВЛИ</h2>
		<p>Для начала выберите свой сервер</p>
		<Button variant="outline-light" className="mt-3" onClick={() => setModalShow(true)}>Выбрать сервер</Button>
		
		<ServerSelectModal show={modalShow} handleClose={() => setModalShow(false)} />
		</Container>
		</div>
		);
	};
	
	export default Main;
\end{lstlisting}

ProfilPage.js
\begin{lstlisting}[language=C++]
	import React, { useContext, useState, useEffect } from 'react';
	import { Container, Row, Col, Button, Form, Image } from 'react-bootstrap';
	import './ProfilPage.css';
	import ChangePasswordModal from "../components/modals/ChangePasswordModal";
	import { Context } from '../index';
	import { check, updateNickname } from '../http/userAPI';
	import { deleteUserItem, fetchUserItems } from '../http/ItemAPI';
	import { useNavigate } from 'react-router-dom';
	import { DEVICE_ROUTE } from '../utils/consts';
	
	const Profile = () => {
		const { user } = useContext(Context);
		const [showChangePasswordModal, setShowChangePasswordModal] = useState(false);
		const [nicknames, setNicknames] = useState({
			ru3: '',
			ru2: '',
			ru1: '',
			eu1: '',
			us1: ''
		});
		const [items, setItems] = useState([]);
		const [userData, setUserData] = useState({ id: 0, email: '', role: '' });
		const navigate = useNavigate();
		
		useEffect(() => {
			const fetchData = async () => {
				try {
					const user = await check();
					
					if (user) {
						const { userId, email, role } = user;
						setUserData({ id: userId, email, role });
						
						console.log("Fetched user data:", user);
						
						// Получение предметов пользователя после установки userData
						const userItems = await fetchUserItems(userId);
						setItems(userItems);
						
						setNicknames({
							ru3: user.ru3 || '',
							ru2: user.ru2 || '',
							ru1: user.ru1 || '',
							eu1: user.eu1 || '',
							us1: user.us1 || ''
						});
					} else {
						console.error('Ошибка аутентификации пользователя');
					}
				} catch (error) {
					console.error('Ошибка при загрузке данных пользователя:', error);
				}
			};
			
			fetchData();
		}, []);
		
		const handleOpenChangePasswordModal = () => setShowChangePasswordModal(true);
		const handleCloseChangePasswordModal = () => setShowChangePasswordModal(false);
		
		const handleInputChange = (event) => {
			const { id, value } = event.target;
			setNicknames((prevState) => ({
				...prevState,
				[id]: value
			}));
		};
		
		const handleSaveNicknames = async () => {
			try {
				await updateNickname(userData.id, nicknames);
			} catch (error) {
				console.error('Ошибка при сохранении никнеймов:', error);
			}
		};
		
		const handleDeleteItem = async (itemId) => {
			try {
				await deleteUserItem(itemId);
				setItems(items.filter(item => item.id !== itemId));
			} catch (error) {
				console.error('Ошибка при удалении предмета:', error);
			}
		};
		
		return (
		<>
		<div className="profile-background"></div>
		<Container className="profile-container">
		<h1 className="profile-title">Профиль пользователя</h1>
		<Row className="profile-row">
		<Col md={6}>
		<h4>Игровой никнейм:</h4>
		<Form>
		{Object.keys(nicknames).map((key) => (
			<Form.Group className="mb-3" controlId={key} key={key}>
			<Form.Label>{key.toUpperCase().replace('_', '-')}:</Form.Label>
			<Form.Control 
			type="text" 
			placeholder="Никнейм..." 
			value={nicknames[key]} 
			onChange={handleInputChange} 
			/>
			</Form.Group>
			))}
		<Button variant="primary" onClick={handleSaveNicknames}>Сохранить изменения</Button>
		</Form>
		</Col>
		<Col md={6}>
		<h4>Ваш email: {userData.email}</h4>
		<Button variant="dark" className="mb-3" onClick={handleOpenChangePasswordModal}>Изменить пароль</Button>
		<h4>Роль: {userData.role}</h4>
		</Col>
		</Row>
		<hr />
		<Row>
		<h4>Выставленные товары:</h4>
		{items.map(item => (
			<React.Fragment key={item.id}>
			<Row className="h-100" style={{ marginRight: 0, marginLeft: 0 }}>
			<Col md={3} className="h-100 d-flex align-items-center justify-content-center" style={{ paddingRight: 0, paddingLeft: 0 }}>
			<Image
			src={process.env.REACT_APP_API_URL + item.item.img}
			fluid
			style={{ maxHeight: '100%', width: 'auto' }}
			onClick={() => navigate(DEVICE_ROUTE + '/' + item.id)}
			/>
			</Col>
			<Col md={6} className="d-flex align-items-center">
			<div>
			<div>{"Название: " + item.item.name}</div>
			<div>{"Цена за единицу: " + item.price}</div>
			<div>{"Количество: " + item.quantity}</div>
			<div>{"Сервер: " + item.server}</div>
			</div>
			</Col>
			<Col md={3} className="d-flex align-items-center justify-content-center">
			<Button variant="danger" onClick={() => handleDeleteItem(item.id)}>Удалить</Button>
			</Col>
			</Row>
			<hr style={{ margin: '10px 0' }} />
			</React.Fragment>
			))}
		</Row>
		<ChangePasswordModal
		show={showChangePasswordModal}
		handleClose={handleCloseChangePasswordModal}
		/>
		</Container>
		</>
		);
	};
	
	export default Profile;
\end{lstlisting}

Shop.js
\begin{lstlisting}[language=C++]
	import React, { useContext, useEffect, useState } from "react";
	import { Button, Col, Container, Form, ListGroup, Row } from "react-bootstrap";
	import TypeBar from "../components/TypeBar";
	import ClassBar from "../components/ClassBar";
	import ItemList from "../components/ItemList";
	import { observer } from "mobx-react-lite";
	import { Context } from "../index";
	import { fetchClasses, fetchItem, fetchStorageItems, fetchTypes, searchItemByName } from "../http/ItemAPI";
	import Pages from "../components/Pages";
	import '../App.css';
	import SellItemModal from "../components/modals/SellItemModal";
	import { useNavigate, useParams } from "react-router-dom";
	
	const Shop = observer(() => {
		const {user, device} = useContext(Context);
		const navigate = useNavigate();
		const { serverId } = useParams();
		const [showModal, setShowModal] = useState(false);
		
		const handleSellItem = async () => {
			setShowModal(false);
		};
		
		useEffect(() => {
			const fetchData = async () => {
				try {
					const typesData = await fetchTypes();
					device.setTypes(typesData);
					
					const classesData = await fetchClasses();
					device.setClasses(classesData);
					
					const storageItemsData = await fetchStorageItems(serverId, device.selectedType.id, device.selectedClass.id, device.page, 5);
					device.setItems(storageItemsData.rows);
					device.setTotalCount(storageItemsData.count);
				} catch (error) {
					console.error('Ошибка при загрузке данных:', error);
				}
			};
			
			fetchData();
			
			const intervalId = setInterval(fetchData, 5000);
			
			return () => clearInterval(intervalId);
		}, [device.selectedType, device.selectedClass, device.page, serverId]);
		
		return (
		<Container>
		<Row className="mt-2">
		<Col md={3}>
		<TypeBar />
		</Col>
		<Col md={9}>
		<ClassBar />
		<ItemList serverId={serverId} />
		<Pages />
		</Col>
		</Row>
		<Button variant="primary" onClick={() => setShowModal(true)} style={{ position: 'fixed', bottom: '20px', right: '20px' }}>
		<i className="fas fa-plus"></i> Выставить на продажу
		</Button>
		<SellItemModal show={showModal} handleClose={() => setShowModal(false)} handleSell={handleSellItem} serverId={serverId} />
		</Container>
		);
	});
	
	export default Shop;
\end{lstlisting}

Auth.js
\begin{lstlisting}[language=C++]
	import React, { useContext, useState } from "react";
	import { Button, Card, Col, Container, Row } from "react-bootstrap";
	import { Form } from "react-bootstrap";
	import { NavLink, useLocation, useNavigate } from "react-router-dom";
	import { LOGIN_ROUTE, MAIN_ROUTE, REGISTRATION_ROUTE } from "../utils/consts";
	import { login, registration } from "../http/userAPI";
	import { observer } from "mobx-react-lite";
	import { Context } from "../index";
	
	const Auth = observer(() => {
		const {user} = useContext(Context)
		const location = useLocation()
		const history = useNavigate()
		const isLogin = location.pathname === LOGIN_ROUTE
		const [email, setEmail] = useState('')
		const [password, setPassword] = useState('')
		
		const click = async () => {
			try {
				let data;
				if (isLogin) {
					data = await login(email, password)
				} else {
					data = await registration(email, password)
				}
				user.setUser(user)
				user.setIsAuth(true)
				history(MAIN_ROUTE)
			} catch(e) {
				alert(e.response.data.message)
			}
		}
		
		return (
		<Container 
		className="d-flex justify-content-center align-items-center"
		style={{height: window.innerHeight - 54}}
		>
		<Card style={{width: 600}} className="p-5">
		<h2 className="m-auto">{isLogin ? "Авторизация" : "Регистрация"}</h2>
		<Form className="d-flex flex-column">
		<Form.Control
		className="mt-4"
		placeholder="Введите ваш email..."
		value={email}
		onChange={e => setEmail(e.target.value)}
		/>
		<Form.Control
		className="mt-3"
		placeholder="Введите ваш пароль..."
		value={password}
		onChange={e => setPassword(e.target.value)}
		type="password"
		/>
		<Row className="d-flex justify-content-between align-items-center mt-3">
		{isLogin ? 
			<Col className="pl-0">
			Нет аккаунта? <NavLink to={REGISTRATION_ROUTE}>Зарегистрируйтесь!</NavLink>
			</Col>
			:
			<Col className="pl-0">
			Есть аккаунт? <NavLink to={LOGIN_ROUTE}>Войдите!</NavLink>
			</Col>
		}
		{isLogin ? 
			<Col className="d-flex justify-content-end pr-0">
			<Button variant="outline-success" onClick={click}>Войти</Button>
			</Col>
			:
			<Col className="d-flex justify-content-end pr-0">
			<Button variant="outline-success" onClick={click}>Зарегистрироваться</Button>
			</Col>
		}
		</Row>
		</Form>
		</Card>
		</Container>
		);
	});
	
	export default Auth;
\end{lstlisting}

ItemItem.js
\begin{lstlisting}[language=C++]
	import React, { useState } from 'react';
	import { Card, Col, Image, Button, Row, Form } from 'react-bootstrap';
	import { useNavigate } from 'react-router-dom';
	import { DEVICE_ROUTE } from '../utils/consts';
	
	const ItemItem = ({ device, serverId }) => {
		const navigate = useNavigate();
		const [isBuying, setIsBuying] = useState(false);
		const formattedServerId = serverId.serverId.replace('-', '');
		
		const userNickname = device.user[formattedServerId];
		
		const handleBuyClick = () => {
			setIsBuying(true);
		};
		
		const handleCancelClick = () => {
			setIsBuying(false);
		};
		
		return (
		<Col md={12}>
		<Card id='cardItem' style={{ cursor: 'pointer', marginBottom: '10px', padding: '10px', fontSize: '14px' }} border="light">
		{!isBuying ? (
			<>
			<Row className="h-100" style={{ marginRight: 0, marginLeft: 0 }}>
			<Col md={3} className="h-100 d-flex align-items-center justify-content-center" style={{ paddingRight: 0, paddingLeft: 0 }}>
			<Image
			src={process.env.REACT_APP_API_URL + device.item.img}
			fluid
			style={{ maxHeight: '100%', width: 'auto' }}
			onClick={() => navigate(DEVICE_ROUTE + '/' + device.item.id)}
			/>
			</Col>
			<Col md={6} className="d-flex align-items-center">
			<div>
			<div>{"Название: " + device.item.name}</div>
			<div>{"Цена за единицу: " + device.price}</div>
			<div>{"Количество: " + device.quantity}</div>
			</div>
			</Col>
			<Col md={3} className="d-flex align-items-center justify-content-center">
			<Button variant="primary" onClick={handleBuyClick}>Купить</Button>
			</Col>
			</Row>
			<hr style={{ margin: '10px 0' }} />
			<Row className="d-flex justify-content-center">
			<div>{"  " + userNickname}</div>
			</Row>
			</>
			) : (
			<>
			<Row className="h-100" style={{ marginRight: 0, marginLeft: 0 }}>
			<Col md={12}>
			<div style={{ marginBottom: '5px' }}>Скопируйте сообщение ниже:</div>
			<Form.Control
			as="textarea"
			rows={3}
			defaultValue={`@${userNickname} Привет. Хочу купить: "${device.item.name}" за ${device.price} рублей (somarket)`}
			style={{ fontSize: '14px' }}
			/>
			</Col>
			</Row>
			<hr style={{ margin: '10px 0' }} />
			<Row className="d-flex justify-content-center">
			<Button variant="outline-danger" size="sm" onClick={handleCancelClick} style={{ width: 'auto', margin: '5 auto' }}>Закрыть</Button>
			</Row>
			</>
			)}
		</Card>
		</Col>
		);
	};
	
	export default ItemItem;
\end{lstlisting}

itemAPI.js
\begin{lstlisting}[language=C++]
	import { $authHost, $host } from "./index";
	
	export const createType = async (type) => {
		const {data} = await $authHost.post('api/type', type);
		return data;
	};
	
	export const fetchTypes = async () => {
		const {data} = await $host.get('api/type');
		return data;
	};
	
	export const createClass = async (classe) => {
		const {data} = await $authHost.post('api/class', classe);
		return data;
	};
	
	export const fetchClasses = async () => {
		const {data} = await $host.get('api/class');
		return data;
	};
	
	export const createItems = async (item) => {
		const {data} = await $authHost.post('api/item', item);
		return data;
	};
	
	export const fetchItem = async (typeId, classId, page, limit = 5) => {
		const {data} = await $host.get('api/item', {
			params: { typeId, classId, page, limit }
		});
		return data;
	};
	
	export const fetchOnItem = async (id) => {
		const {data} = await $host.get('api/item/item/' + id);
		return data;
	};
	
	export const searchItemByName = async (name) => {
		try {
			const {data} = await $authHost.get(`api/item/search/${name}`);
			return data;
		} catch (error) {
			console.error('Ошибка при поиске предмета:', error);
			throw error;
		}
	};
	
	export const addItemToStorage = async ({ userId, itemId, price, quantity, typeId, classId, server }) => {
		try {
			const { data } = await $authHost.post('api/item/storage/add', {
				userId,
				itemId,
				price,
				quantity,
				typeId,
				classId,
				server
			});
			return data;
		} catch (error) {
			console.error('Ошибка при добавлении предмета в корзину:', error);
			throw error;
		}
	};
	
	export const fetchStorageItems = async (serverId, typeId, classId, page, limit = 10) => {
		try {
			const { data } = await $host.get(`api/item/storage/items/${serverId}`, {
				params: { typeId, classId, page, limit }
			});
			return data;
		} catch (error) {
			console.error('Ошибка при загрузке предметов из хранилища:', error);
			throw error;
		}
	};
	
	export const fetchUserItems = async (userId) => {
		try {
			const { data } = await $host.get('api/item/user/items', {
				params: { userId }
			});
			return data;
		} catch (error) {
			console.error('Ошибка при загрузке предметов пользователя:', error);
			throw error;
		}
	};
	
	export const deleteUserItem = async (id) => {
		const { data } = await $authHost.delete('api/item/' + id);
		return data;
	};
	
	export const fetchAmmoInfos = async (itemId) => {
		try {
			const { data } = await $host.get(`api/ammo_infos/${itemId}`);
			return data;
		} catch (error) {
			console.error('Error fetching ammo_infos:', error);
			throw error;
		}
	};
\end{lstlisting}

userAPI.js
\begin{lstlisting}[language=C++]
	import { $authHost, $host } from "./index";
	import { jwtDecode } from "jwt-decode";
	
	export const registration = async (email, password) => {
		const {data} = await $host.post('api/user/registration', {email, password, role: 'ADMIN'});
		localStorage.setItem('token', data.token);
		return jwtDecode(data.token);
	};
	
	export const login = async (email, password) => {
		const {data} = await $host.post('api/user/login', {email, password});
		localStorage.setItem('token', data.token);
		return jwtDecode(data.token);
	};
	
	export const check = async () => {
		try {
			const token = localStorage.getItem('token');
			
			if (!token) {
				throw new Error('Token is not available');
			}
			
			const decodedToken = jwtDecode(token);
			const currentTime = Date.now() / 1000;
			
			if (decodedToken.exp < currentTime) {
				throw new Error('Token has expired');
			}
			
			const { data } = await $authHost.get('api/user/auth');
			
			const userData = {
				token: data.token,
				userId: decodedToken.id,
				email: decodedToken.email,
				role: decodedToken.role,
				ru1: decodedToken.ru1,
				ru2: decodedToken.ru2,
				ru3: decodedToken.ru3,
				eu1: decodedToken.eu1,
				us1: decodedToken.us1,
			};
			
			localStorage.setItem('token', userData.token);
			return userData;
		} catch (error) {
			console.error('Authentication check failed:', error);
			return null;
		}
	};
	
	export const changePassword = async (currentPassword, newPassword) => {
		const {data} = await $authHost.post('/api/user/change-password', { currentPassword, newPassword });
		return data;
	};
	
	export const updateNickname = async (userId, nicknames) => {
		try {
			await $authHost.put('api/user/nicknames', { userId, nicknames });
		} catch (error) {
			console.error('Ошибка при обновлении никнейма:', error);
			throw error;
		}
	};
\end{lstlisting}

\ifВКР{
\newpage
\addcontentsline{toc}{section}{На отдельных листах (CD-RW в прикрепленном конверте)}
\begin{center}
\textbf{Место для диска}
\end{center}
}\fi
