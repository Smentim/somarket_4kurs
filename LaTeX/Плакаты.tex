\appendix{Представление графического материала}

Графический материал, выполненный на отдельных листах,
изображен на рисунках А.1--А.\arabic{числоПлакатов}.
\setcounter{числоПлакатов}{0}

\renewcommand{\thefigure}{А.\arabic{figure}} % шаблон номера для плакатов

\begin{landscape}

\begin{плакат}
    \includegraphics[width=0.82\linewidth]{плакат1.png}
    \заголовок{Сведения о ВКРБ}
    \label{pl1:image}      
\end{плакат}

\begin{плакат}
    \includegraphics[width=0.82\linewidth]{плакат2.png}
    \заголовок{Цели и задачи разработки}
    \label{pl2:image}      
\end{плакат}

\begin{плакат}
    \includegraphics[width=0.82\linewidth]{плакат3.png}
    \заголовок{Концептуальная диаграмма сущность-связь}
    \label{pl3:image}      
\end{плакат}

\begin{плакат}
    \includegraphics[width=0.82\linewidth]{плакат4.png}
    \заголовок{Диаграмма прецедентов}
    \label{pl4:image}      
\end{плакат}

\begin{плакат}
	\centering
	\includegraphics[width=0.82\linewidth]{images/плакат5}
	\caption{Диаграмма компонентов}
	\label{pl5:image}
\end{плакат}

\begin{плакат}
	\centering
	\includegraphics[width=0.82\linewidth]{images/плакат6}
	\caption{Диаграмма размещения}
	\label{pl6:image}
\end{плакат}

\begin{плакат}
	\centering
	\includegraphics[width=0.82\linewidth]{images/плакат7}
	\caption{Карта маршрутов}
	\label{pl7:image}
\end{плакат}


\end{landscape}
